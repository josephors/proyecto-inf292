\clearpage
\section{Análisis de Resultados}
\label{sec:analisis}

En esta sección se presenta un análisis completo de los resultados obtenidos, incluyendo relaciones entre variables, escalabilidad temporal, verificación visual mediante calendarios, y diagnóstico detallado de las instancias infactibles.

\subsection{Función Objetivo vs. Tamaño del Problema}

El valor de la función objetivo muestra una relación lineal fuerte con el tamaño del problema (medido como trabajadores \(\times\) días), como se observa en la Figura~\ref{fig:objetivo_tamano}.

\begin{figure}[h]
  \centering
  \includegraphics[width=0.85\textwidth]{../Entrega 2/analisis/graficos/objetivo_vs_tamano.png}
  \caption{Relación entre el valor objetivo y el tamaño del problema. La línea punteada muestra la tendencia lineal ajustada.}
  \label{fig:objetivo_tamano}
\end{figure}

\paragraph{Observaciones:}
\begin{itemize}
  \item La ecuación de tendencia es \(y = 8.78x - 75.80\), con un coeficiente de correlación muy alto (\(r > 0.99\)).
  \item El valor objetivo crece proporcionalmente con el número de trabajadores y días disponibles.
  \item Las instancias small muestran menor variabilidad, mientras que las large presentan mayor dispersión debido a sus mayores grados de libertad.
\end{itemize}

\subsection{Tiempos de Resolución y Escalabilidad}

El tiempo de resolución muestra un crecimiento aproximadamente cuadrático con el tamaño del problema, como se ilustra en la Figura~\ref{fig:tiempos_tamano}.

\begin{figure}[h]
  \centering
  \includegraphics[width=0.85\textwidth]{../Entrega 2/analisis/graficos/tiempos_vs_tamano.png}
  \caption{Tiempo de resolución en función del tamaño del problema. Los puntos marcados con X representan instancias infactibles.}
  \label{fig:tiempos_tamano}
\end{figure}

\paragraph{Análisis de escalabilidad:}
\begin{itemize}
  \item \textbf{Small:} Tiempo promedio 0.052s. Resolución prácticamente instantánea.
  \item \textbf{Medium:} Tiempo promedio 0.197s. Factor de crecimiento \(\approx 3.8\times\) respecto a small.
  \item \textbf{Large:} Tiempo promedio 0.240s (solo instancias factibles). Factor de crecimiento \(\approx 4.6\times\) respecto a small.
  \item Las instancias infactibles (12 y 15) se detectan más rápidamente (~0.1--0.2s) sin necesidad de explorar completamente el espacio de soluciones.
\end{itemize}

\subsection{Calendarios Visuales de Asignaciones}

Para las instancias small (1--5), se generaron calendarios visuales que permiten verificar el cumplimiento de restricciones y la distribución de carga. Un ejemplo se muestra en la Figura~\ref{fig:resumen_calendarios}.

\begin{figure}[h]
  \centering
  \includegraphics[width=0.95\textwidth]{../Entrega 2/analisis/graficos/calendarios/resumen_todas_instancias.png}
  \caption{Resumen visual de asignaciones para instancias small. El color indica el número de turnos asignados por trabajador por día (0=blanco, 1=amarillo claro, 2=rojo).}
  \label{fig:resumen_calendarios}
\end{figure}

\paragraph{Verificación visual:}
\begin{itemize}
  \item \textbf{R1 (Cobertura):} Los calendarios detallados muestran que cada turno requerido está cubierto exactamente.
  \item \textbf{R3 (Máximo 2 turnos/día):} No se observan celdas con más de 2 turnos.
  \item \textbf{Distribución de carga:} La carga está relativamente balanceada entre trabajadores, con variaciones debidas a diferencias en disponibilidad individual.
\end{itemize}

\subsection{Diagnóstico de Infactibilidad}
El script de análisis calcula métricas clave:
\begin{itemize}
  \item Demanda total agregada y demanda por turno (mañana, tarde, noche).
  \item Capacidad teórica diaria (R3) y utilización porcentual.
  \item Disponibilidad total y por turno (R2), incluyendo porcentaje de slots disponibles.
  \item Conflictos potenciales turno noche \(\rightarrow\) turno mañana (R4), midiendo la proporción de demanda de noche seguida de demanda de mañana.
  \item Límite de carga \(L_i\) (R6) comparado contra la carga mínima necesaria por trabajador.
  \item Heurística de tensión de fines de semana (R5) evaluando si, en bloques de tres fines, la demanda agregada obliga a activaciones consecutivas.
\end{itemize}

\subsection*{Métricas Clave de Instancias Infactibles}
En la Tabla \ref{tab:infactibles} se muestra un resumen numérico para las instancias infactibles.

\begin{table}[h]
  \centering
  \caption{Métricas clave instancias infactibles.}
  \label{tab:infactibles}
  \begin{tabular}{lrrrrr}
    \hline
    Instancia & Demanda Total & Slots Disp. (R2) & $L_i$ & Carga Mín. Req. & Ratio Conflictos (R4) \\
    \hline
    12 & 1003 & 2418 & 11 & 20 & 0.90 \\
    15 & 1194 & 2952 & 9  & 16 & 0.76 \\
    \hline
  \end{tabular}
\end{table}

\paragraph{Interpretación:}
\begin{itemize}
  \item \textbf{R6 (Límite de carga).} En ambos casos la carga mínima requerida por trabajador (demanda total / trabajadores) excede claramente \(L_i=\lfloor 2H/3 \rfloor\). Esto basta para generar infactibilidad: aunque la disponibilidad sea alta, la restricción de tope global bloquea asignaciones suficientes.
  \item \textbf{R4 (Noche\(\rightarrow\)Mañana).} La proporción de pares potencialmente conflictivos es elevada (\(>0.75\)), reduciendo la reutilización eficiente de trabajadores entre días consecutivos y amplificando la presión sobre R6.
  \item \textbf{R2 (Disponibilidad).} No es la causa dominante (porcentaje disponible > 90\%), lo que confirma que el cuello de botella no está en ausencia de disposición sino en las restricciones estructurales.
  \item \textbf{R5 (Fines de semana).} La heurística indica tensión: la demanda agregada en secuencias de tres fines de semana fuerza casi el uso continuo de muchos trabajadores, chocando con la prohibición de tres activaciones consecutivas.
\end{itemize}

\subsection*{Principales Causas Identificadas}
\begin{enumerate}
  \item \textbf{(R6) Límite de carga insuficiente:} \(L_i\) demasiado bajo respecto de la densidad de demanda. La instancia exige más turnos por trabajador que el máximo permitido.
  \item \textbf{(R4) Alta fricción noche\(\rightarrow\)mañana:} La secuencia de demanda nocturna seguida de demanda matinal limita combinaciones factibles y aumenta fragmentación de asignaciones.
  \item \textbf{(R5) Tensión en fines de semana:} Bloques de tres fines con demanda elevada reducen opciones, especialmente cuando muchos trabajadores ya alcanzan el tope \(L_i\).
\end{enumerate}

\subsection*{Recomendaciones para Mejora de Factibilidad}
Para futuras generaciones de instancias o ajuste de parámetros:
\begin{itemize}
  \item Aumentar \(L_i\) o reducir demanda total si la carga mínima \(> L_i\).
  \item Redistribuir demanda para bajar la concentración de pares noche\(\rightarrow\)mañana (p.ej. suavizar picos en noches consecutivas).
  \item Asegurar que la demanda en bloques de tres fines de semana no obligue activación casi universal (balanceo inter-semanal).
  \item Mantener disponibilidad por turno \(\ge 1.2\times\) la demanda para conservar holgura.
\end{itemize}

\subsection*{Conclusión}
La infactibilidad de las instancias 12 y 15 surge de la interacción entre un límite de carga agresivo (R6) y patrones de demanda que elevan los conflictos (R4 y R5). Ajustar \(L_i\) y redistribuir demanda temporal ofrece la vía más directa para recuperar factibilidad sin modificar la estructura central del modelo.

