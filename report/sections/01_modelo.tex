\clearpage
\section{Modelo Matemático}

\subsection*{Conjuntos e Índices}
\begin{itemize}
  %Conjunto de trabajadores
  \item[$T$:] Conjunto de trabajadores, indexado por \( i \in T \).
  %Conjunto de Días del horizonte de planificación 
  \item[$D$:] Conjunto de días del horizonte de planificación, definido como
  \[
      D = \{1, 2, \ldots, H\},
  \]
  donde \( H \) representa el número total de días. Cada día se indexa por \( j \in D \).
  %Conjunto de turnos disponibles (mañana, tarde, noche)
  \item[$S$:] Conjunto de turnos disponibles por día, indexado por \( t \in S \).
\end{itemize}

\subsection*{Parámetros}
\begin{itemize}
  
  % Nivel de disposición del trabajador
  \item[$c_{i,j,t}$:] Nivel de disposición del trabajador \( i \) para realizar el turno \( t \) del día \( j \), donde \( c_{i,j,t} \in \{0,1,\dots,10\} \). Un valor de 0 indica que el trabajador no puede realizar ese turno.
  
  % Demanda de trabajadores en día
  \item[$r_{j,t}$:] Cantidad requerida de trabajadores para cubrir el turno \( t \) del día \( j \), con \( r_{j,t} \in \mathbb{N} \).
  
  \item[$a_{i,j,t}$:] indica si el trabajador  \( i \)  está disponible para realizar el turno \( t \) del día \( j \), definida como:
  \[
  a_{i,j,t} =
  \begin{cases}
  1 & \text{si } c_{i,j,t} > 0, \\
  0 & \text{si } c_{i,j,t} = 0.
  \end{cases}
  \]
  

  \item[$W_k$:] Conjunto de días correspondientes al fin de semana de la semana \( k \), utilizado para controlar la asignación de turnos en sábados y domingos. Se define como:
  \[
  W_k = \{ \text{s\'abado de la semana } k,\ \text{domingo de la semana } k \}, \quad \text{para } k = 1, \dots, K
  \]
  donde el número total de semanas \( K \) se calcula como:
  \[
  K = \left\lfloor \frac{H}{7} \right\rfloor
  \]
  siendo \( H \) el número total de días del horizonte de planificación.

  \item[$L_i$:] Límite máximo de turnos que el trabajador \( i \) puede ser asignado durante todo el horizonte de planificación. Este parámetro permite controlar la carga total de trabajo por persona, promoviendo equidad y evitando sobreasignaciones. Se puede definir como:
  \[
  L_i = \left\lfloor \frac{2H}{3} \right\rfloor
  \]
  donde \( H \) es el número total de días del horizonte.

\end{itemize}

\subsection*{Variables de decisión}
\begin{itemize}
  \item[$x_{i,j,t}$:] Variable binaria que indica si el trabajador \( i \) es asignado al turno \( t \) del día \( j \), definida como:
  \[
  x_{i,j,t} =
  \begin{cases}
    1, & \text{si el trabajador } i \text{ es asignado al turno } t \text{ del día } j, \\
    0, & \text{en caso contrario}.
  \end{cases}
  \]

  \item[$y_{i,k}$:] Variable binaria que indica si el trabajador \( i \) realiza al menos un turno durante el fin de semana correspondiente a la semana \( k \). Se define como:
  \[
  y_{i,k} =
  \begin{cases}
  1 & \text{si el trabajador } i \text{ realiza al menos un turno en el fin de semana } k, \\
  0 & \text{en caso contrario}.
  \end{cases}
  \]
  
\end{itemize}

\clearpage
\subsection*{Función Objetivo}

\textbf{Función Objetivo:} Maximizar la disposición total del personal asignado, favoreciendo la asignación de turnos a quienes tienen mayor disposición declarada.

\[
\max Z = \sum_{i \in T} \sum_{j \in D} \sum_{t \in S} x_{i,j,t} \cdot c_{i,j,t}
\]


\subsection*{Restricciones}

El modelo considera las siguientes restricciones para asegurar la factibilidad operativa y el bienestar del personal:

\begin{itemize}
  \item[\textbf{(R1)}] \textbf{Cobertura de turnos:} Cada turno debe ser cubierto exactamente por la cantidad requerida de trabajadores.
  \[
  \sum_{i \in T} x_{i,j,t} = r_{j,t} \quad \forall j \in D,\ \forall t \in S
  \]

  \item[\textbf{(R2)}] \textbf{Disponibilidad individual:} Un trabajador solo puede ser asignado a un turno si tiene disposición estrictamente positiva para realizarlo.
  \[
  x_{i,j,t} \le a_{i,j,t} \quad \forall i \in T,\ \forall j \in D,\ \forall t \in S
  \]

  \item[\textbf{(R3)}] \textbf{Máximo de dos turnos por día:} Ningún trabajador puede realizar más de dos turnos en un mismo día.
  \[
  \sum_{t \in S} x_{i,j,t} \le 2 \quad \forall i \in T,\ \forall j \in D
  \]

  \item[\textbf{(R4)}] \textbf{Descanso entre noche y mañana:} No se permite asignar a un trabajador al turno de noche de un día y al turno de mañana del día siguiente.\footnote{En instancias \textit{small}, el conjunto de turnos $S$ solo incluye \texttt{día} y \texttt{noche}, por lo que esta restricción no aplica.}
  \[
  x_{i,j,\text{noche}} + x_{i,j+1,\text{mañana}} \le 1 \quad \forall i \in T,\ \forall j \in D \setminus \{H\}
  \]

  \item[\textbf{(R5)}] \textbf{No trabajar tres fines de semana consecutivos:} Un trabajador no puede estar asignado a turnos en tres fines de semana seguidos.
  \[
  y_{i,k} + y_{i,k+1} + y_{i,k+2} \le 2 \quad \forall i \in T,\ \forall k \in \{1, \dots, K-2\}
  \]

  \item[\textbf{(R5.1)}] \textbf{Definición de actividad en fin de semana:} La variable $y_{i,k}$ se activa si el trabajador realiza al menos un turno en el fin de semana $k$.
  \[
  y_{i,k} \ge x_{i,d,t} \quad \forall d \in W_k,\ \forall t \in S
  \]
  \[
  y_{i,k} \le \sum_{d \in W_k} \sum_{t \in S} x_{i,d,t} \quad \forall i \in T,\ \forall k \in \{1, \dots, K\}
  \]

  \item[\textbf{(R6)}] \textbf{Límite de carga total por trabajador:} Cada trabajador puede ser asignado como máximo a \( L_i \) turnos en todo el horizonte de planificación.
  \[
  \sum_{j \in D} \sum_{t \in S} x_{i,j,t} \le L_i \quad \forall i \in T
  \]

  \item[\textbf{(R7)}] \textbf{Naturaleza de las variables:} Se definen como binarias las variables de asignación y de actividad en fin de semana.
  \[
  x_{i,j,t} \in \{0,1\}, \quad y_{i,k} \in \{0,1\}
  \]
  
\end{itemize}