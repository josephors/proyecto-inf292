\clearpage
\section{Modelo Matemático}

\subsection*{\textbf{Conjuntos, Parámetros y Variables}}

\vspace{0.2cm}
\textbf{Conjuntos e índices:}
\begin{itemize}
  \item[$T$:] Trabajadores, indexados por \( i \in T \).
  \item[$D$:] Días del horizonte de planificación, \( D = \{1, 2, \ldots, H\} \), con \( j \in D \).
  \item[$S$:] Turnos disponibles por día, indexados por \( t \in S \).
  \item[$W_k$:] Días del fin de semana de la semana \( k \), donde \( W_k = \{\text{s\'abado}, \text{domingo}\} \), y \( K = \left\lfloor \frac{H}{7} \right\rfloor \).
\end{itemize}

\vspace{0.2cm}
\textbf{Parámetros:}
\begin{itemize}
  \item[$c_{i,j,t}$:] Disposición de \( i \) para el turno \( t \) del día \( j \), con valores entre 0 y 10.
  \item[$r_{j,t}$:] Demanda requerida de trabajadores para el turno \( t \) del día \( j \).
  \item[$a_{i,j,t}$:] Disponibilidad binaria, definida como:
  

\[
  a_{i,j,t} =
  \begin{cases}
  1 & \text{si } c_{i,j,t} > 0, \\
  0 & \text{si } c_{i,j,t} = 0.
  \end{cases}
  \]


\end{itemize}

\vspace{0.2cm}
\textbf{Variables de decisión:}
\begin{itemize}
  \item[$x_{i,j,t}$:] Binaria. 1 si el trabajador \( i \) es asignado al turno \( t \) del día \( j \).
  \item[$y_{i,k}$:] Binaria. 1 si el trabajador \( i \) realiza al menos un turno en el fin de semana \( k \).
\end{itemize}

\subsection*{\textbf{Función Objetivo}}



\[
\max Z = \sum_{i \in T} \sum_{j \in D} \sum_{t \in S} c_{i,j,t} \cdot x_{i,j,t}
\]



\vspace{0.2cm}
\textbf{Interpretación:} Se maximiza la disposición total del personal asignado, favoreciendo la asignación de turnos a quienes tienen mayor disposición declarada. Esto permite una planificación más eficiente y respetuosa con las preferencias individuales.

\subsection*{\textbf{Restricciones}}

\begin{align*}
&\text{(R1)} & \sum_{i \in T} x_{i,j,t} = r_{j,t} &\quad \forall j \in D,\; t \in S \\
&\text{(R2)} & x_{i,j,t} \le a_{i,j,t} &\quad \forall i,j,t \\
&\text{(R3)} & \sum_{t \in S} x_{i,j,t} \le 2 &\quad \forall i,j \\
&\text{(R4)} & x_{i,j,\text{noche}} + x_{i,j+1,\text{mañana}} \le 1 &\quad \forall i,\; j < H \\
&\text{(R5)} & y_{i,k} + y_{i,k+1} + y_{i,k+2} \le 2 &\quad \forall i,\; k \le K-2 \\
&\text{(R5.1)} & x_{i,d,t} \le y_{i,k} &\quad \forall d \in W_k,\; t \in S \\
& & y_{i,k} \le \sum_{d \in W_k} \sum_{t \in S} x_{i,d,t} &\quad \forall i,\; k \\
&\text{(R6)} & x_{i,j,t},\; y_{i,k} \in \{0,1\} &\quad \forall i,j,t,\; k
\end{align*}

\subsection*{\textbf{Explicación de las Restricciones}}

\textbf{R1:} Garantiza que cada turno sea cubierto exactamente por la cantidad requerida de trabajadores.
\vspace{0.2cm}

\textbf{R2:} Asegura que solo se asignen turnos a trabajadores disponibles.
\vspace{0.2cm}

\textbf{R3:} Limita la carga diaria de cada trabajador a un máximo de dos turnos.
\vspace{0.2cm}

\textbf{R4:} Impide asignaciones consecutivas de noche y mañana para respetar los descansos.
\vspace{0.2cm}

\textbf{R5:} Evita que un trabajador trabaje tres fines de semana seguidos, promoviendo el equilibrio.
\vspace{0.2cm}

\textbf{R5.1:} Define cuándo se activa la variable \( y_{i,k} \), vinculándola con la actividad real en fines de semana.

\vspace{0.2cm}
\textbf{R6:} Define la naturaleza binaria de las variables de decisión, asegurando que las asignaciones sean discretas.


\subsection*{\textbf{Notas explicativas}}

\begin{itemize}
  \item \textbf{Eliminación de tope de carga:} Se descarta la restricción sobre el total de turnos por trabajador planteada en la entrega 1. La carga se regula mediante R3, R5 y la función objetivo.
  
  \item \textbf{R5 y R5.1:} R5 limita la frecuencia de trabajo en fines de semana; R5.1 define cuándo se activa esa condición. Juntas aseguran control lógico y preciso.

  \item \textbf{R4 en instancias pequeñas:} Si \( S \) solo incluye \texttt{día} y \texttt{noche}, R4 no aplica. Debe considerarse al validar instancias.
\end{itemize}

