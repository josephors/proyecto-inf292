\clearpage
\section{Modelo Matemático}

\subsection*{Objetivo}
Maximizar la \textbf{disposición total del personal asignado}, asegurando simultáneamente el cumplimiento de las \textbf{restricciones operativas} y de \textbf{bienestar del personal}, tales como la cobertura de turnos, la carga laboral diaria, los descansos adecuados y la equidad en la asignación de fines de semana.

La función objetivo se define como la suma de las disposiciones individuales de los trabajadores asignados a cada turno, buscando así una planificación que favorezca la satisfacción del personal sin comprometer la continuidad del servicio.

\subsection*{Conjuntos e Índices}
\begin{itemize}
  \item[$T$:] Conjunto de trabajadores, indexado por \( i \in T \).
  \item[$D$:] Conjunto de días del horizonte de planificación, definido como
  \[
      D = \{1, 2, \ldots, H\},
  \]
  donde \( H \) representa el número total de días. Cada día se indexa por \( j \in D \).
  \item[$S$:] Conjunto de turnos disponibles por día, indexado por \( t \in S \).
\end{itemize}

\subsection*{Parámetros}
\begin{itemize}
  \item[$H$:] Número total de días del horizonte de planificación. Define la cardinalidad del conjunto \( D \).
  
  \item[$c_{i,j,t}$:] Nivel de disposición del trabajador \( i \) para realizar el turno \( t \) del día \( j \), donde \( c_{i,j,t} \in \{0,1,\dots,10\} \). Un valor de 0 indica que el trabajador no puede realizar ese turno.
  
  \item[$r_{j,t}$:] Cantidad requerida de trabajadores para cubrir el turno \( t \) del día \( j \), con \( r_{j,t} \in \mathbb{N} \).
  
  \item[$W_j$:] Indicador binario que vale 1 si el día \( j \) corresponde a un fin de semana (sábado o domingo), y 0 en caso contrario.
\end{itemize}

\subsection*{Variables de decisión}
\begin{itemize}
  \item[$x_{i,j,t}$:] Variable binaria que indica si el trabajador \( i \) es asignado al turno \( t \) del día \( j \), definida como:
  \[
  x_{i,j,t} =
  \begin{cases}
    1, & \text{si el trabajador } i \text{ es asignado al turno } t \text{ del día } j, \\
    0, & \text{en caso contrario}.
  \end{cases}
  \]
\end{itemize}

\clearpage
\subsection*{Función Objetivo}

El objetivo del modelo es \textbf{maximizar la disposición total del personal asignado}, es decir, la suma de los niveles de disposición de los trabajadores que efectivamente son asignados a turnos. Esto se representa mediante la siguiente función:

\[
\max Z = \sum_{i \in T} \sum_{j \in D} \sum_{t \in S} x_{i,j,t} \cdot c_{i,j,t}
\]

donde:
\begin{itemize}
  \item \( x_{i,j,t} \) indica si el trabajador \( i \) es asignado al turno \( t \) del día \( j \),
  \item \( c_{i,j,t} \) representa la disposición del trabajador \( i \) para trabajar en esa combinación día-turno.
\end{itemize}

Esta función busca asignar turnos a quienes tienen mayor disposición, promoviendo así una planificación que favorezca el bienestar del personal sin comprometer la cobertura requerida.


\subsection*{Restricciones}

El modelo considera las siguientes restricciones para asegurar la factibilidad operativa y el bienestar del personal:

\begin{itemize}
  \item[\textbf{(R1)}] \textbf{Cobertura de turnos:} Cada turno debe ser cubierto exactamente por la cantidad requerida de trabajadores.
  \[
  \sum_{i \in T} x_{i,j,t} = r_{j,t} \quad \forall j \in D,\ \forall t \in S
  \]

  \item[\textbf{(R2)}] \textbf{Disponibilidad individual:} Un trabajador solo puede ser asignado a un turno si tiene disposición estrictamente positiva para realizarlo.
  \[
  x_{i,j,t} \leq 
  \begin{cases}
  1, & \text{si } c_{i,j,t} > 0 \\
  0, & \text{si } c_{i,j,t} = 0
  \end{cases}
  \quad \forall i \in T,\ \forall j \in D,\ \forall t \in S
  \]
  \item[\textbf{(R3)}] \textbf{Máximo de dos turnos por día:} Ningún trabajador puede realizar más de dos turnos en un mismo día.
  \[
  \sum_{t \in S} x_{i,j,t} \leq 2 \quad \forall i \in T,\ \forall j \in D
  \]

  \item[\textbf{(R4)}] \textbf{Descanso entre noche y mañana:} No se permite asignar a un trabajador al turno de noche de un día y al turno de mañana del día siguiente.
  \[
  x_{i,j,\text{Noche}} + x_{i,j+1,\text{Mañana}} \leq 1 \quad \forall i \in T,\ \forall j \in D \setminus \{H\}
  \]

  \item[\textbf{(R5)}] \textbf{No trabajar tres fines de semana consecutivos:} Un trabajador no puede estar asignado a turnos en tres fines de semana seguidos.
  \[
  w_{i,k} + w_{i,k+1} + w_{i,k+2} \leq 2 \quad \forall i \in T,\ \forall k \in \{1, \dots, K-2\}
  \]
\end{itemize}

\vspace{0.5em}

\noindent
Para la restricción (R5), se define una variable auxiliar \( w_{i,k} \in \{0,1\} \) que indica si el trabajador \( i \) realiza al menos un turno durante el fin de semana \( k \). Esta variable se relaciona con las asignaciones reales mediante:
\[
w_{i,k} \geq x_{i,d,t} \quad \forall d \in \text{días del fin de semana } k,\ \forall t \in S
\]