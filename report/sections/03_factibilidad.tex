\clearpage
\section{Análisis de Factibilidad}

El análisis de factibilidad evalúa si las instancias generadas permiten soluciones que cumplan las restricciones del modelo matemático. El generador no fuerza factibilidad, pero incorpora controles estructurales que favorecen la viabilidad operativa.

\subsection*{Controles incorporados}

\begin{itemize}[leftmargin=1em]
    \item \textbf{Cobertura por turno (R1):} La demanda por turno se ajusta según la cantidad de trabajadores con disposición positiva, evitando requerimientos imposibles de cubrir.

    \item \textbf{Capacidad diaria (R3):} La suma de la demanda por día se limita a un máximo de dos turnos por trabajador, respetando la restricción de carga diaria.

    \item \textbf{Carga total (R6):} Aunque el parámetro \( L_i = \left\lfloor \frac{2H}{3} \right\rfloor \) no se usa directamente en el generador, se respeta de forma indirecta al limitar la demanda diaria y la carga agregada por trabajador. Esto favorece la factibilidad sin imponerla explícitamente.

    \item \textbf{Fines de semana (R5):} Se controla que la demanda agregada en sábados y domingos no exceda la capacidad global del equipo.
\end{itemize}

\subsection*{Chequeo automatizado}

Se implementaron tres verificaciones automáticas:

\begin{itemize}[leftmargin=1em]
    \item \textbf{Check A:} Demanda por turno \(\leq\) disponibilidad efectiva.
    \item \textbf{Check B:} Demanda diaria  \(\leq\) capacidad total.
    \item \textbf{Check C:} Demanda en fines de semana  \(\leq\) capacidad agregada.
\end{itemize}

Estas verificaciones permiten detectar y corregir sobrecargas antes de resolver el modelo.

\subsection*{Restricciones delegadas al modelo}

Las siguientes restricciones dependen de decisiones de asignación y no pueden verificarse en la etapa de generación o chequeo. Por ello, se delegan a la resolución del modelo:

\begin{itemize}[leftmargin=1em]
    \item \textbf{(R2)} Asignación solo si el trabajador tiene disposición positiva.
    \item \textbf{(R4)} Prohibición de turno noche seguido de mañana.
    \item \textbf{(R5)} No trabajar tres fines de semana consecutivos.
\end{itemize}

\subsection*{Observaciones}

Las instancias \textit{small} tienden a ser factibles. En \textit{medium} y \textit{large}, la factibilidad depende de la relación entre demanda y disposición. La aleatoriedad permite explorar escenarios diversos, incluyendo casos límite útiles para validar el modelo.

\clearpage
