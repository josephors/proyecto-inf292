\clearpage
\section{Análisis de Factibilidad}

El análisis de factibilidad busca evaluar si las instancias generadas permiten encontrar soluciones viables que cumplan todas las restricciones del modelo. Dado que el generador produce datos de forma aleatoria, es posible que algunas instancias resulten infactibles, especialmente en los casos de mayor tamaño o con parámetros extremos.

\subsection*{Factores que afectan la factibilidad}

\begin{itemize}
  \item \textbf{Demanda vs. oferta:} La demanda por turno y día se genera mediante una distribución normal truncada, con media proporcional al número de trabajadores disponibles. Sin embargo, al tratarse de una generación aleatoria, pueden producirse casos en que la demanda total supere la capacidad máxima del personal, especialmente si muchos trabajadores tienen baja o nula disposición.
  
  \item \textbf{Disposición del personal:} La disposición \( c_{i,j,t} \sim \mathcal{U}\{0,10\} \) puede generar combinaciones en las que no haya suficientes trabajadores disponibles para ciertos turnos, lo que impide cumplir la cobertura requerida.

  \item \textbf{Restricciones de bienestar:} Condiciones como el máximo de dos turnos por día, la prohibición de turnos noche-mañana consecutivos y la limitación de fines de semana trabajados pueden restringir aún más el espacio de soluciones factibles, especialmente en instancias con pocos trabajadores o alta demanda.
\end{itemize}

\subsection*{Observaciones preliminares}

Durante la revisión manual de las instancias generadas, se observó que:

\begin{itemize}
  \item Las instancias \textit{small} tienden a ser factibles con mayor frecuencia, dado que las restricciones son menos exigentes y el número de turnos es menor.
  \item En instancias \textit{medium} y \textit{large}, la factibilidad depende fuertemente de la relación entre la demanda generada y la disposición del personal. En algunos casos, se detectaron instancias infactibles debido a una combinación de alta demanda y baja disposición.
  \item La aleatoriedad en la generación permite evaluar el modelo en escenarios diversos, incluyendo casos límite que son útiles para validar la robustez del enfoque.
\end{itemize}

\subsection*{Justificación del diseño del generador}

El uso de distribuciones probabilísticas permite simular situaciones realistas y variadas, incluyendo tanto instancias factibles como infactibles. Esto es intencional, ya que permite:

\begin{itemize}
  \item Evaluar la capacidad del modelo para detectar infactibilidades.
  \item Analizar cómo las restricciones afectan la existencia de soluciones.
  \item Ajustar parámetros del generador para controlar la dificultad de las instancias.
\end{itemize}

Este análisis será complementado en la Entrega 2 mediante la resolución efectiva de las instancias y el estudio de los resultados obtenidos.

\clearpage