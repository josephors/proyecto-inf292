\clearpage
\section{Análisis de Factibilidad}

El análisis de factibilidad evalúa si las instancias generadas permiten soluciones que cumplan las restricciones del modelo. Dado que el generador utiliza datos aleatorios, pueden aparecer casos infactibles, especialmente en instancias grandes o con parámetros extremos. Por ello, el diseño del generador considera explícitamente los factores que más afectan la factibilidad, con el objetivo de producir instancias razonables y representativas.

\subsection*{Factores que influyen}
\begin{itemize}[leftmargin=1em]
    \item \textbf{Cobertura por turno:} Para cada día y turno, la demanda se ajusta según la capacidad disponible, es decir, el número de trabajadores que pueden cubrir ese turno según su disposición. Esto evita generar demandas imposibles de cubrir.

    \item \textbf{Capacidad diaria:} La suma de la demanda de todos los turnos de un día se limita a la capacidad total teórica del personal, considerando un máximo de turnos por trabajador por día (por ejemplo, dos). Esto previene sobrecargas que violarían restricciones del modelo.

    \item \textbf{Congestión en fines de semana:} Los días sábado y domingo suelen tener menor disponibilidad. El generador controla que la demanda agregada en estos días no supere la capacidad global del equipo, considerando el límite total de turnos por trabajador en el horizonte.
\end{itemize}

\subsection*{Observaciones}
\begin{itemize}[leftmargin=1em]
  \item Las instancias \textit{small} tienden a ser factibles, mientras que en \textit{medium} y \textit{large} la factibilidad depende de la relación entre demanda y disposición.
  \item El diseño del generador evita casos triviales o claramente infactibles, lo que permite concentrarse en instancias relevantes para el modelo.
  \item La aleatoriedad en la generación permite explorar una variedad de escenarios, incluyendo casos límite útiles para validar la robustez del enfoque.
  \item El control sobre parámetros como la media de demanda, los umbrales de disposición y el límite de turnos por trabajador permite ajustar la dificultad de las instancias.
\end{itemize}

\subsection*{Justificación del generador}
El generador fue diseñado para producir instancias que respeten las restricciones estructurales del modelo, en particular aquellas relacionadas con la capacidad de cobertura y la carga máxima por trabajador. El uso de distribuciones probabilísticas permite simular situaciones realistas y diversas, facilitando el análisis de factibilidad y el estudio del comportamiento del modelo bajo distintas condiciones. Además, al incorporar directamente el límite de turnos por trabajador (\( L_i = \left\lfloor \frac{2H}{3} \right\rfloor \)), se garantiza el cumplimiento de la restricción (R6) desde la etapa de generación.

Este análisis se complementará en la Entrega 2 con la resolución efectiva de las instancias y el estudio de resultados.
\clearpage