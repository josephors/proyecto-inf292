\section*{Resumen Ejecutivo}

\vspace{0.5em}

\noindent
Este informe presenta la formulación de un modelo de \textbf{programación lineal entera binaria} para resolver el problema de \textit{asignación de turnos en una clínica de atención integral}, considerando tanto la cobertura operativa como el bienestar del personal. El modelo incorpora restricciones clave como la carga máxima diaria, descansos entre turnos críticos y equidad en fines de semana trabajados, maximizando la disposición declarada por los trabajadores.

\vspace{0.5em}

\noindent
Además, se implementó un \textbf{generador de instancias sintéticas} que permite evaluar el modelo en distintos tamaños de problema (\textit{small}, \textit{medium}, \textit{large}), generando datos de demanda y disposición de forma aleatoria bajo distribuciones controladas.

\vspace{0.5em}

\noindent
El lector encontrará en este documento la definición formal del modelo, junto con una descripción del generador de instancias, sentando las bases para su posterior resolución computacional y análisis en la segunda entrega.