\clearpage
\section{Generador de Instancias y Factibilidad}

El generador fue implementado en \textbf{Python} y se encuentra en:

\begin{center}
\texttt{Entrega\_2\_Grupo2\_OPTI\_SJ/Generador\_1\_Grupo2\_OPTI\_SJ.py}
\end{center}

Este script genera \textbf{cinco instancias por cada tamaño} (\textit{small}, \textit{medium}, \textit{large}), siguiendo los rangos establecidos en el enunciado. Las instancias se almacenan en carpetas separadas dentro de:

\begin{center}
\texttt{Entrega\_2\_Grupo2\_OPTI\_SJ/data/}
\end{center}

El código fuente completo está disponible en el repositorio de GitHub \cite{generador2025}. Además, se incluye un script de chequeo que valida condiciones de factibilidad y genera resúmenes estadísticos por instancia.

\subsection*{\textbf{Lógica de Generación}}

Se utilizó la semilla \texttt{42} para asegurar reproducibilidad. El generador elimina carpetas previas para evitar residuos y garantizar un entorno limpio en cada ejecución.

\begin{itemize}
  \item Se generan aleatoriamente los días, trabajadores y turnos según el tipo de instancia:
  \begin{itemize}
    \item \textit{Small}: 5--7 días, 5--15 trabajadores, turnos \texttt{día} y \texttt{noche}.
    \item \textit{Medium}: 7--14 días, 15--45 trabajadores, turnos \texttt{mañana}, \texttt{tarde}, \texttt{noche}.
    \item \textit{Large}: 14--28 días, 45--90 trabajadores, mismos turnos que \textit{medium}.
  \end{itemize}

  \item Para cada día y turno, se genera una demanda \(r_{j,t}\) con distribución normal truncada en cero, ajustada por:
  \begin{itemize}
    \item Disponibilidad efectiva (trabajadores con disposición positiva).
    \item Capacidad máxima diaria (\( \text{max\_turnos\_dia} \times \text{trabajadores} \)).
  \end{itemize}

  \item Las disposiciones \(c_{i,j,t} \sim \mathcal{U}\{0,10\}\) se asignan por trabajador, día y turno.

  \item Cada instancia se guarda en dos formatos:
  \begin{itemize}
    \item \texttt{.json}: incluye metadatos, demanda y disposiciones.
    \item \texttt{.csv}: contiene las disposiciones individuales por fila.
  \end{itemize}
\end{itemize}

\newpage
\subsection*{\textbf{Condiciones de Factibilidad en la Generación}}

El generador fue diseñado para producir instancias estructuralmente válidas, es decir, con datos completos, consistentes y compatibles con el modelo matemático. Esto implica que cada instancia contiene trabajadores, días, turnos, disposiciones y demandas bien definidas, sin errores lógicos ni violaciones formales.

Además, se incorporan controles que reflejan las restricciones del modelo desde la etapa de generación:

\begin{itemize}
  \item \textbf{R1 (Cobertura):} La demanda por turno se ajusta según la cantidad de trabajadores disponibles con disposición positiva.
  \item \textbf{R3 (Carga diaria):} Se limita la suma de turnos por día, escalando la demanda si excede la capacidad total del equipo.
  \item \textbf{R5 (Fines de semana):} Se evita congestión en sábado y domingo, respetando la carga global del equipo.
\end{itemize}

Aunque no se incluye una restricción explícita sobre el total de turnos por trabajador (R6), la carga se regula indirectamente mediante R3, R5 y la función objetivo. Las instancias \textit{small} tienden a ser factibles, mientras que en \textit{medium} y \textit{large} la factibilidad depende de la relación entre demanda y disposición.

\vspace{0.4cm}
\textbf{Importante:} Que una instancia sea estructuralmente válida no implica que sea resoluble. La factibilidad se evalúa posteriormente mediante el modelo LPE, que determina si existe una asignación que cumpla todas las restricciones. Este enfoque permite estudiar cómo influyen los parámetros generados aleatoriamente en la posibilidad de encontrar soluciones válidas, sin alterar artificialmente las instancias para forzar su resolución.
