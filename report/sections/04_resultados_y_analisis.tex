% VEAN LA PAUTA % VEAN LA PAUTA % VEAN LA PAUTA % VEAN LA PAUTA 


% MATI Y FER

% SE HARAN CARGO DE ESTA PARTE PARA EL PPT Y EL REPORTE

% VEAN LOS RESUTADOS Y ANALISIS DENTRO DE ./ENTREGA 2/ANALISIS Y LOS RESULTADOS A ANALIZAR ESTAN EN ./ENTREGA 2/RESULTADOS


% PARA EL ANALISIS HAY UNOS SCRIPTS DENTRO DE  ./ENTREGA 2/ANALISIS:

% generar_calendarios QUE GUARDA EN ./ENTREGA 2/ANALISIS/GRAFICOS/CALENDARIOS LA ASIGNACIONES DE LAS 5 INSTANCIAS SMALL OJO. HAY UN README PARA ENTENDER CMO SE LEE

% generar_graficos_objetivo QUE GENERA EN ./ENTREGA 2/ANALISIS/GRAFICOS LOS GRAFOS correlacion_matriz, objetivo_promedio_tipo, objetivo_trabajadores_dias, objetivo_vs_tamano, SON PUROS GRAFOS LINEALES, ASI QUE USEN LO QUE LES SIRVA, NO CREO QUE SEA NECESARIO USAR TODO Y EL DE CORRELACION TAMPOCO


% LO MISMO CON  generar_graficos_tiempos, QUE GENERA EN ./ENTREGA 2/ANALISIS/GRAFICOS LOS GRAFOS RESTANTES, LO MISMO QUE PARA LO ANTERIOR, NO ES NECESARIO USAR TODOS LOS GRAFOS USEN LO MAS INTERESANTE COMO TIMPO POR TAMAÑO Y WEAS ASI MAS SUSTANCIOSAS.

% SEAN CONSISES EN EL REPORTE Y LA PPT, ENTIENDAN A GRANDES RASGOS, PERO ENTIENDAN Y RECUERDEN QUE USTEDES HABLARAN DE ESTO EN LA PRESENTACION!

% CUALQUIER DUDA PREGUNTAME POR EL GRUPO :)

% VEAN LA PAUTA % VEAN LA PAUTA % VEAN LA PAUTA % VEAN LA PAUTA 

\clearpage
\section{Resultados y Análisis}

\subsection*{\textbf{Resumen general de la ejecución}}

El modelo fue resuelto usando la herramienta \textbf{LPSolve} a través de un script que procesó las quince instancias generadas (\textit{small}, \textit{medium} y \textit{large}).   

El archivo \texttt{ejecutar\_solver\_batch.py} ejecutó el solver para cada instancia y registró los resultados en formato JSON.   

En total, se obtuvieron 13 soluciones óptimas y 2 instancias infactibles. Los tiempos de ejecución fueron muy bajos (del orden de milisegundos), confirmando la eficiencia del modelo en instancias pequeñas y medianas.


\begin{table}[H]
\centering
\begin{tabular}{lcccc}
\toprule
\textbf{Tamaño} & \textbf{Total} & \textbf{Óptimas} & \textbf{Infactibles} & \textbf{Tiempo promedio (s)} \\
\midrule
Small & 5 & 5 & 0 & 0.000613 \\
Medium & 5 & 5 & 0 & 0.009846 \\
Large & 5 & 3 & 2 & 0.208286\\
\midrule
\textbf{Total} & 15 & 13 & 2 & --- \\
\bottomrule
\end{tabular}
\caption{Resumen de resultados globales por tipo de instancia.}
\end{table}

\vspace{0.4cm}

De la tabla anterior se puede observar que las instancias \textit{small} y \textit{medium} fueron completamente factibles y alcanzaron el óptimo en menos de 0.05 segundos.  
En cambio, dos de las instancias \textit{large} resultaron
 infactibles, lo cual se analizará más adelante.

 \newpage
\subsection*{\textbf{(a) Análisis de la función objetivo}}

La función objetivo maximiza la disposición 
total del personal asignado.  
A medida que el tamaño de la instancia crece (más trabajadores y días), el valor objetivo aumenta de manera proporcional al número de asignaciones posibles, manteniendo así, la coherencia del modelo.

\begin{figure}[H]
\centering
\includegraphics[width=0.7\textwidth]{../Entrega_2_Grupo2_OPTI_SJ/analisis/plots/objetivo/objetivo_promedio_tipo.png}
\caption{Promedio del valor objetivo por tipo de instancia.}
\end{figure}

En la Figura anterior se puede apreciar un crecimiento proporcional al tamaño de la instancia del valor objetivo promedio al pasar de \textit{small} a \textit{large}, esto es coherente con el aumento de la 
demanda y la disponibilidad total de turnos.  
Lo que demuestra que el modelo logra mantener la consistencia de asignaciones al priorizar trabajadores con mayor disposición.
\newpage

\begin{figure}[H]
\centering
\includegraphics[width=0.7\textwidth]{../Entrega_2_Grupo2_OPTI_SJ/analisis/plots/objetivo/objetivo_vs_tamano.png}
\caption{Comportamiento del valor objetivo en función del tamaño (número de variables).}
\end{figure}

En la Figura 2 se puede observar que la relación entre el valor objetivo y el número de variables muestra una tendencia ascendente
con una leve dispersión.  
Esto refleja que el solver sigue encontrando soluciones óptimas, pero la complejidad combinatoria puede introducir ligeras variaciones en la calidad o el tiempo de resolución.  
En resumen, el modelo mantiene su desempeño y escalabilidad dentro del rango planificado.

\subsection*{\textbf{(b) Análisis de infactibilidad}}
Que una solución sea infactible (o que el modelo es infeasible) significa que no existe ninguna combinación de valores que cumpla todas las restricciones del modelo al mismo tiempo.

En total, dos instancias de tipo \textit{large} fueron declaradas infactibles.  
Esto puede deberse principalmente a la interacción entre las restricciones de cobertura (R1), carga diaria (R3) y descanso (R4).  
En ciertas combinaciones aleatorias de disponibilidad y demanda, 
puede ocurrir que la cantidad de trabajadores dispuestos para cubrir un turno crítico sea insuficiente o que las reglas de descanso hagan imposible una cobertura completa.

\begin{itemize}
  \item En particular, R4 (\textit{noche $\rightarrow$ mañana}) puede reducir la disponibilidad efectiva cuando los turnos nocturnos son muy numerosos.
  \item R3 (\textit{máx. 2 turnos por día}) restringe aún más la flexibilidad, limitando las posibles reasignaciones. \end{itemize}

Es importante destacar que esto no representa un error del modelo, sino una propiedad esperable en escenarios donde 
la demanda supera la capacidad disponible.  
De hecho, la existencia de instancias infactibles es útil para validar la robustez del generador y del modelo de optimización.
\subsection*{\textbf{(c) Visualización tipo calendario para instancias pequeñas}}

En las instancias pequeñas, el modelo permite visualizar fácilmente la asignación diaria por trabajador.  
La siguiente figura muestra un ejemplo de calendario generado por el script \texttt{Generar\_calendarios\_2\_Grupo2\_OPTI\_SJ.py}, donde cada celda indica los turnos asignados a cada empleado durante la semana.
\begin{figure}[H]
\centering
\includegraphics[width=0.8\textwidth]{../Entrega_2_Grupo2_OPTI_SJ/analisis/plots/calendarios/calendario_instancia_1.png}
\caption{Ejemplo de calendario de asignaciones para una instancia \textit{small}.}
\end{figure}
En este calendario, se puede observar que las asignaciones 
respetan las restricciones R3 y R4: ningún trabajador realiza más de dos turnos diarios ni tiene secuencias noche–mañana(teniendo en cuenta que no aplica para instancias small).  
El modelo logra tener una distribución equilibrada de la carga laboral entre los trabajadores disponibles, maximizando la disposición total.

\newpage
\subsection*{\textbf{Análisis de tiempos de resolución}}
Finalmente, se evaluó el tiempo de resolución promedio y su relación con el tamaño de las instancias 
y la cantidad de variables involucradas.

\begin{figure}[H]
\centering
\includegraphics[width=0.8\textwidth]{../Entrega_2_Grupo2_OPTI_SJ/analisis/plots/tiempos/tiempos_promedio_tipo.png}
\caption{Tiempo promedio de resolución por tipo de instancia.}
\end{figure}
En el gráfico se puede observar que el tiempo crece de forma 
proporcional al pasar de instancias pequeñas a grandes.  
También se puede ver, que incluso en el caso \textit{large}, el tiempo promedio se mantiene bajo (0.2 s), lo que confirma la eficiencia del modelo implementado.

\newpage
\begin{figure}[H]
\centering
\includegraphics[width=0.8\textwidth]{../Entrega_2_Grupo2_OPTI_SJ/analisis/plots/tiempos/tiempos_vs_variables.png}
\caption{Tiempos de resolución en función del número de variables.}
\end{figure}
En la figura anterior se puede observar que, aunque el tiempo de resolución 
tiende a aumentar con el número de variables,
este crecimiento no es abrupto ni exponencial, lo que indica que el modelo escala de forma controlada 
y que el solver maneja de forma eficiente los casos con mayor tamaño.  
En general, el comportamiento temporal es coherente con la naturaleza del modelo, 
donde el aumento de combinaciones posibles no afecta de manera drástica la capacidad de resolución gracias 
a la estructura binaria y a las restricciones acotadas del problema.


\vspace{0.5cm}
Al comparar los resultados de tiempo entre los tres tamaños, se evidencia que la diferencia entre \textit{small} y \textit{medium} es mínima,
mientras que el salto hacia \textit{large} representa el aumento esperado en complejidad, pero sin llegar a comprometer la factibilidad ni el rendimiento.  
Esto demuestra que el modelo es estable y eficiente, incluso cuando la cantidad de trabajadores, días o turnos se multiplica varias veces.

\newpage
\vspace{0.5cm}
\subsection*{\textbf{Conclusión general}}
En conclusión, el modelo propuesto demuestra ser sólido, eficiente y coherente con los objetivos planteados.  
A lo largo de las distintas instancias, se ha podido confirmar que:
\begin{itemize}
  \item El modelo mantiene la escalabilidad, incrementando el valor de la función objetivo de forma proporcional al tamaño de la instancia.
  \item Los tiempos de resolución son reducidos incluso en casos de gran tamaño, evidenciando una implementación eficiente en \textbf{LPSolve}.
  \item Las instancias infactibles se explican por condiciones realistas de sobrecarga de demanda, lo cual logra validar la robustez del modelo.
  \item Las soluciones factibles respetan todas las restricciones definidas, reflejando una distribución equilibrada de la carga laboral.
\end{itemize}

Además, el análisis conjunto de los resultados 
y los tiempos de resolución permite concluir que el modelo no solo optimiza correctamente la asignación de personal, 
sino que también lo hace con un costo computacional bastante bajo.  
Esto sugiere que la formulación matemática podría escalar a entornos reales 
más amplios sin comprometer la eficiencia del proceso.

En resumen, el sistema de generación, modelamiento y resolución cumple con los criterios de calidad, desempeño y factibilidad requeridos,
demostrando así, un comportamiento consistente, interpretable y aplicable en contextos reales de planificación de turnos.
