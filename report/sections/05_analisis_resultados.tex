\section{Análisis de resultados}
\label{sec:analisis-resultados}

\subsection{Función objetivo vs tamaño}
La Figura~\ref{fig:obj-por-tamano} resume la relación entre el valor de la función objetivo y el tamaño de instancia. Observamos un crecimiento casi lineal con el número de variables: la correlación con el tamaño agregado es muy alta (\emph{1.000} con tamaño, \emph{0.977} con número de trabajadores y \emph{0.904} con días). La tendencia estimada es $\hat{y}=8.78\,x-75.43$.

\begin{figure}[H]
	\centering
	\includegraphics[width=0.95\textwidth]{../Entrega 2/analisis/graficos/objetivo_vs_tamano.png}
	\caption{Función objetivo vs tamaño de instancia (small/medium/large).}
	\label{fig:obj-por-tamano}
\end{figure}

Complementamos con vistas alternativas: descomposición por trabajadores y días, promedio por tipo y matriz de correlación (Figuras~\ref{fig:obj-trab-dias} y \ref{fig:correlacion}).

\begin{figure}[H]
	\centering
	\begin{subfigure}{0.49\textwidth}
		\centering
		\includegraphics[width=\linewidth]{../Entrega 2/analisis/graficos/objetivo_trabajadores_dias.png}
	\end{subfigure}\hfill
	\begin{subfigure}{0.49\textwidth}
		\centering
		\includegraphics[width=\linewidth]{../Entrega 2/analisis/graficos/objetivo_promedio_tipo.png}
	\end{subfigure}
	\caption{Descomposición de la función objetivo por trabajadores/días y promedio por tipo.}
	\label{fig:obj-trab-dias}
\end{figure}

\begin{figure}[H]
	\centering
	\includegraphics[width=0.6\textwidth]{../Entrega 2/analisis/graficos/correlacion_matriz.png}
	\caption{Matriz de correlación entre objetivo y descriptores.}
	\label{fig:correlacion}
\end{figure}

\subsection{Infactibilidades: instancias 12 y 15}
Ambas instancias resultan infactibles. La causa principal es la interacción entre (R1) \emph{cobertura exacta} y (R4) \emph{no noche→mañana}. Aunque la disponibilidad global supera el 90\%, la estructura de turnos y bloqueos secuenciales tras la noche genera cuellos de botella que impiden satisfacer \emph{exactamente} todas las demandas. Se detalla caso a caso en el cuaderno de análisis y en \texttt{Entrega 2/analisis/factibilidad.md}.

\subsection{Soluciones gráficas (instancias small)}
Para las instancias \textit{small} generamos calendarios estilo ``trabajador × día'' que permiten auditar visualmente (R1) y (R3). La Figura~\ref{fig:resumen-calendarios} muestra un resumen comparativo; en cada celda se indica el número de turnos asignados (0, 1 o 2). También disponemos de los cinco calendarios individuales.

\begin{figure}[H]
	\centering
	\includegraphics[width=0.98\textwidth]{../Entrega 2/analisis/graficos/calendarios/resumen_todas_instancias.png}
	\caption{Resumen de asignaciones (instancias small). Cada subgráfico: \#trabajadores × \#días; color y número indican \#turnos por trabajador y día.}
	\label{fig:resumen-calendarios}
\end{figure}

\noindent\textbf{Nota sobre R4 en small.} En estas instancias solo existen los turnos \emph{día} y \emph{noche}; no hay turno \emph{mañana}. Por ello, R4 (no noche→mañana) aplica literalmente a \textit{medium/large}. De requerirse, puede activarse una variante ``no noche→día'' para \textit{small}.