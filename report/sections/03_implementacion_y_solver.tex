\newpage
\section{Metodología de Implementación del Solver}

Esta sección describe la arquitectura del nuevo solver desarrollado usando 
\texttt{lpsolve55}, el flujo de resolución implementado y el mecanismo utilizado
para procesar todas las instancias y almacenar sus resultados. 
Todo el desarrollo se encuentra documentado en \texttt{Entrega\_2\_Grupo2\_OPTI\_SJ/},
incluyendo los archivos:

\vspace{0.4cm}
\noindent \texttt{Ejecutor\_Solver\_lpsolve\_2\_Grupo2\_OPTI\_SJ.py} y \texttt{Solver\_lpsolve\_2\_Grupo2\_OPTI\_SJ.py}.

\vspace{0.4cm}
\subsection*{\textbf{Arquitectura del Sistema}}

El solver se organiza en dos componentes principales:

\begin{itemize}
    \item \textbf{\texttt{Solver\_lpsolve\_2\_Grupo2\_OPTI\_SJ.py}}: Implementa el modelo matemático directamente en lp\_solve 5.5 y resuelve una única instancia.
    \item \textbf{\texttt{Ejecutor\_Solver\_lpsolve\_2\_Grupo2\_OPTI\_SJ.py}}: Recorre todas las carpetas de instancias, ejecuta el solver para cada archivo y genera los reportes correspondientes.
\end{itemize}

Esta separación permite mantener el código modular: un archivo se enfoca en la definición del problema y el otro en la ejecución masiva y la recolección de métricas.

\vspace{0.4cm}
\subsection*{\textbf{Implementación de la Función Objetivo}}

El solver utiliza \textbf{Programación Lineal Entera}, modelada con \texttt{lp\_solve 5.5}.  
La función objetivo maximiza la disposición total:

\[
\max Z = \sum_{i \in T} \sum_{j \in D} \sum_{t \in S} c_{ijt} \cdot x_{ijt}
\]

\begin{lstlisting}[caption=Implementación de la Función Objetivo en lpsolve55]
# F.O.: Maximizar la suma de disposiciones
obj = [0.0] * nvars
for i in range(num_trabajadores):
    for j in range(num_dias):
        for t in range(num_turnos):
            idx = index_x(i, j, t) - 1
            obj[idx] = disposicion[i][j][t]

lpsolve('set_obj_fn', lp, obj)
\end{lstlisting}

\vspace{0.4cm}
\subsection*{\textbf{Implementación de las Restricciones}}

El solver implementa las restricciones R1–R5.1 tal como se definieron en la especificación del proyecto:

\begin{itemize}
    \item \textbf{R1: Cobertura exacta de demanda por turno.}  
    Cada turno de cada día debe cubrir exactamente la demanda de trabajadores. Esto se implementa sumando todas las variables \(x_{ijt}\) correspondientes a un turno y agregando una restricción de igualdad con la demanda:

\begin{lstlisting}[caption=Restricción R1 en lpsolve55]
# R1: Cobertura exacta de demanda
for j in range(num_dias):
    for t in range(num_turnos):
        row = fila_cero()
        for i in range(num_trabajadores):
            row[index_x(i,j,t)-1] = 1.0
        lpsolve('add_constraint', lp, row, '=', demanda[j][t])
\end{lstlisting}

    \item \textbf{R2: Prohibición de asignar turnos con disposición 0.}  
    Si un trabajador no está disponible para un turno específico, se fija \(x_{ijt} = 0\):

\begin{lstlisting}[caption=Restricción R2 en lpsolve55]
# R2: Prohibicion de turnos no disponibles
for i in range(num_trabajadores):
    for j in range(num_dias):
        for t in range(num_turnos):
            if disposicion[i][j][t] == 0:
                row = fila_cero()
                row[index_x(i,j,t)-1] = 1.0
                lpsolve('add_constraint', lp, row, '=', 0)
\end{lstlisting}

    \item \textbf{R3: Máximo de 2 turnos por día por trabajador.}  
    Un trabajador no puede estar asignado a más de dos turnos en un mismo día:

\begin{lstlisting}[caption=Restricción R3 en lpsolve55]
# R3: Maximo 2 turnos por dia
for i in range(num_trabajadores):
    for j in range(num_dias):
        row = fila_cero()
        for t in range(num_turnos):
            row[index_x(i,j,t)-1] = 1.0
        lpsolve('add_constraint', lp, row, '<=', 2)
\end{lstlisting}

\newpage
    \item \textbf{R4: Prohibición de turno noche seguido por turno mañana.}  
    Se evita que un trabajador tenga turno noche seguido de turno mañana al día siguiente:

\begin{lstlisting}[caption=Restricción R4 en lpsolve55]
# R4: No noche -> manana consecutiva
for i in range(num_trabajadores):
    for j in range(num_dias-1):
        row = fila_cero()
        row[index_x(i,j,t_noche)-1] = 1.0
        row[index_x(i,j+1,t_man)-1] = 1.0
        lpsolve('add_constraint', lp, row, '<=', 1)
\end{lstlisting}

    \item \textbf{R5: No trabajar 3 fines de semana consecutivos.}  
    Esta restricción utiliza variables binarias auxiliares \(w_{ik}\) que indican si el trabajador \(i\) trabaja durante el fin de semana \(k\). Se asegura que la suma de \(w\) de tres fines consecutivos no exceda 2:

\begin{lstlisting}[caption=Restricción R5 en lpsolve55]
# R5: No trabajar 3 fines de semana consecutivos
for i in range(num_trabajadores):
    for k in range(num_fines-2):
        row = fila_cero()
        row[index_w(i,k)-1] = 1.0
        row[index_w(i,k+1)-1] = 1.0
        row[index_w(i,k+2)-1] = 1.0
        lpsolve('add_constraint', lp, row, '<=', 2)
\end{lstlisting}

\newpage
    \item \textbf{R5.1: Enlace bidireccional entre \(x\) y \(w\).}  
    Cada \(w_{ik}\) se vincula con las variables \(x_{ijt}\) correspondientes a los días del fin de semana \(k\):

\begin{lstlisting}[caption=Restricción R5.1 en lpsolve55]
# R5.1: Enlace x <= w y w <= Sum x
for i in range(num_trabajadores):
    for k, dias_fin in enumerate(fines_de_semana):
        # x <= w
        for dia_idx in dias_fin:
            j = dia_idx-1
            for t in range(num_turnos):
                row = fila_cero()
                row[index_x(i,j,t)-1] = 1.0
                row[index_w(i,k)-1] = -1.0
                lpsolve('add_constraint', lp, row, '<=', 0)
        # w <= Sum x
        row = fila_cero()
        row[index_w(i,k)-1] = 1.0
        for dia_idx in dias_fin:
            j = dia_idx-1
            for t in range(num_turnos):
                row[index_x(i,j,t)-1] -= 1.0
        lpsolve('add_constraint', lp, row, '<=', 0)
\end{lstlisting}

\end{itemize}

\vspace{0.4cm}
\subsection*{\textbf{Proceso de Ejecución Masiva}}

El archivo \texttt{Ejecutor\_Solver\_lpsolve\_2\_Grupo2\_OPTI\_SJ.py} recorre automáticamente los directorios:

\begin{center}
\texttt{data/small/}, \texttt{data/medium/}, \texttt{data/large/}
\end{center}

Para cada archivo JSON:

\begin{enumerate}
    \item Se carga la instancia.
    \item Se construye el modelo en \texttt{lp\_solve 5.5}.
    \item Se ejecuta el solver.
    \item Se mide el tiempo de resolución.
    \item Se clasifica el resultado como óptimo, factible o infactible.
    \item Se guarda un archivo JSON en:
\end{enumerate}

\[
\texttt{Entrega\_2\_Grupo2\_OPTI\_SJ/resultados/<tamaño>/}
\]

\newpage
\vspace{0.4cm}
\subsection*{\textbf{Estructura del Módulo de Métricas}}

El sistema acumula información global para producir un archivo resumen:

\begin{lstlisting}[caption=Estructura de Recolección de Métricas]
resumen_total = {
    "fecha_ejecucion": datetime.now().isoformat(),
    "instancias_procesadas": 0,
    "instancias_optimas": 0,
    "instancias_infactibles": 0,
    "tiempo_total_segundos": 0,
    "resultados_por_tamano": {}
}
\end{lstlisting}

\vspace{0.4cm}
\subsection*{\textbf{Ejemplo de Archivo de Resultados}}

\begin{lstlisting}[language=json, caption=Ejemplo de Resultado JSON]
{
  "id_instancia": 11,
  "tipo": "large",
  "dias": 23,
  "trabajadores": 47,
  "estado": "Optimal",
  "valor_objetivo": 9362.0,
  "tiempo_resolucion_segundos": 0.179936,
  "factible": true,
  "asignaciones": [
    {
      "trabajador": 1,
      "dia": 1,
      "dia_nombre": "domingo_1",
      "turno": "noche",
      "disposicion": 8
    }
  ],
  "fecha_resolucion": "2025-11-09T02:00:57.060042"
}
\end{lstlisting}

\newpage
\vspace{0.4cm}
\subsection*{\textbf{Notas Finales}}

\begin{itemize}
    \item El sistema está diseñado para ser escalable y completamente automatizado.
    \item Todos los archivos están documentados en \texttt{Entrega\_2\_Grupo2\_OPTI\_SJ/solver/}.
    \item Se elimina explícitamente la restricción R6 (tope global de carga).
    \item La ejecución de instancias grandes puede demorar más debido a la naturaleza entera del problema.
    \item Para la utilización del solver es necesario descargar Anaconda, ya que es quien soporta la librería del solver pedido.
\end{itemize}
