% VEAN LA PAUTA % VEAN LA PAUTA % VEAN LA PAUTA 


% CHAKA

% DOCUMENTA EN ./ENTREGA 2/SOLVER/

% LOS ARCHIVOS DE EJECUTAR_SOLVER_BATCH.PY Y SOLUCIONADOR_DEINSTANCIAS_LPSOLVE


% ESTAS A CARGO DE HABLAR SOBRE ESTO Y COMO GUARDAR LOS RESULTADOS DENTRO DE ./ENTREGA 2/RESULTADOS/ SEGUN EL TIPO DE INSTANCIA

% SE CONSISO EN EL REPORTE Y LA PPT, ENTIENDE A GRANDES RASGOS, PERO ENTIENDE, RECUERDA QUE TU HABLARAS DE ESTO EN LA PRESENTACION!

% CUALQUIER DUDA PREGUNTAME POR EL GRUPO : )

% VEAN LA PAUTA % VEAN LA PAUTA % VEAN LA PAUTA 


\documentclass{article}

\usepackage[utf8]{inputenc}
\usepackage[spanish]{babel}
\usepackage{amsmath}
\usepackage{amssymb}
\usepackage{listings}
\usepackage{geometry}
\usepackage{xcolor}

\geometry{a4paper, margin=1in}

\lstset{
    language=Python,
    basicstyle=\footnotesize\ttfamily,
    numbers=left,
    numberstyle=\tiny,
    stepnumber=1,
    numbersep=5pt,
    frame=single,
    showtabs=false,
    showspaces=false,
    showstringspaces=false,
    captionpos=b,
    breaklines=true,
    breakatwhitespace=true
}

\begin{document}

\section{Metodología de Implementación (Solver)}

La implementación se realizó bajo un modelo de \textbf{Programación Lineal Entera} utilizando la librería \textbf{PuLP} con el solver \textbf{Lpsolver}.

\subsubsection{Implementación F.O.}
El objetivo es maximizar la suma de la disposición ($c_{ijt}$) de los trabajadores asignados:
\begin{equation}
    \max Z = \sum_{i \in T} \sum_{j \in D} \sum_{t \in S} x_{ijt} \cdot c_{ijt}
\end{equation}

El código para la definición de esta función en \texttt{PuLP} es el siguiente:

\begin{lstlisting}[caption=Definición de Función Objetivo con PuLP]
# F.O.: Maximizar la suma de disposiciones
modelo += pulp.lpSum(
    x[i][j][t] * disposicion[i][j][t]
    for i in range(num_trabajadores)
    for j in range(num_dias)
    for t in range(num_turnos)
), "Maximizar_Disposicion"
\end{lstlisting}

\subsection{Implementación de la Restricción R5}

La restricción más compleja es la \textbf{R5 (No 3 Fines de Semana Consecutivos)}. Requiere la variable auxiliar $w_{ik}$ y un enlace bidireccional para asegurar coherencia.

\begin{equation}
    w_{i,k} + w_{i,k+1} + w_{i,k+2} \le 2 \quad \forall i \in T, \ \forall k
\end{equation}

\begin{lstlisting}[caption=Lógica de la Restricción R5 (No 3 Fines Consecutivos)]
# Restriccion R5: No trabajar 3 fines de semana consecutivos
if num_fines >= 3:
    # ... (Se incluye R5.1: el enlace bidireccional entre x y w) ...

    for i in range(num_trabajadores):
        for k in range(num_fines - 2):
            # La suma de 3 w's consecutivos no puede superar 2
            modelo += (
                w[i][k] + w[i][k + 1] + w[i][k + 2] <= 2,
                f"No3Fines_{i}_{k}"
            )
\end{lstlisting}

\section{Toma de Datos y Recopilación de Métricas}

\subsection{Proceso de Ejecución}

\begin{lstlisting}[caption=Estructura de Recolección de Métricas Globales]
# Localizar directorios del proyecto
instancias_dir = base_dir / "Entrega 1" / "generador" / "instancias"
resultados_dir = base_dir / "Entrega 2" / "resultados"

tamanos = ["small", "medium", "large"]

# Estructura para acumular estadísticas globales
resumen_total = {
    "fecha_ejecucion": datetime.now().isoformat(),
    "instancias_procesadas": 0,
    "instancias_optimas": 0,
    "instancias_infactibles": 0,
    "tiempo_total_segundos": 0,
    "resultados_por_tamano": {}
}
\end{lstlisting}

\subsection{Ejemplo de Resultado JSON}

\begin{lstlisting}[language=json, caption=Extracto de Resultado JSON (Instancia Large)]
{
  "id_instancia": 11,
  "tipo": "large",
  "dias": 23,
  "trabajadores": 47,
  "estado": "Optimal",
  "valor_objetivo": 9362.0,
  "tiempo_resolucion_segundos": 0.179936,
  "factible": true,
  "asignaciones": [
    {
      "trabajador": 1,
      "dia": 1,
      "dia_nombre": "domingo_1",
      "turno": "noche",
      "disposicion": 8
    },
    {
      "trabajador": 47,
      "dia": 23,
      "dia_nombre": "viernes_3",
      "turno": "manana",
      "disposicion": 8
    }
  ],
  "fecha_resolucion": "2025-11-09T02:00:57.060042"
}
\end{lstlisting}

\end{document}
