\clearpage
\section{Metodología de Implementación del Solver}

Esta sección describe la arquitectura del solver, el flujo de resolución implementado y el mecanismo utilizado para procesar todas las instancias y almacenar sus resultados. Todo el desarrollo se encuentra documentado en \texttt{./Entrega 2/solver/}, incluyendo los archivos \texttt{ejecutar\_solver\_batch.py} y \texttt{solucionador\_de\_instancias\_lpsolve.py}.

\vspace{0.4cm}
\subsection*{\textbf{Arquitectura del Sistema}}

El solver se organiza en dos componentes principales:

\begin{itemize}
    \item \textbf{\texttt{solucionador\_de\_instancias\_lpsolve.py}}: Implementa el modelo matemático en PuLP y resuelve una única instancia.
    \item \textbf{\texttt{ejecutar\_solver\_batch.py}}: Recorre todas las carpetas de instancias, ejecuta el solver para cada archivo y genera los reportes correspondientes.
\end{itemize}

Esta separación permite mantener el código modular: un archivo se enfoca en la definición del problema y el otro en la ejecución masiva y la recolección de métricas.

\vspace{0.4cm}
\subsection*{\textbf{Implementación de la Función Objetivo}}

El solver utiliza \textbf{Programación Lineal Entera}, modelada con la librería \texttt{PuLP} junto al backend \texttt{lpsolve}.  
La función objetivo maximiza la disposición total:

\[
\max Z = \sum_{i \in T} \sum_{j \in D} \sum_{t \in S} c_{ijt} \cdot x_{ijt}
\]

\begin{lstlisting}[caption=Implementación de la Función Objetivo en PuLP]
# F.O.: Maximizar la suma de disposiciones
modelo += pulp.lpSum(
    x[i][j][t] * disposicion[i][j][t]
    for i in range(num_trabajadores)
    for j in range(num_dias)
    for t in range(num_turnos)
), "Maximizar_Disposicion"
\end{lstlisting}

\newpage
\vspace{0.4cm}
\subsection*{\textbf{Restricción R5: No trabajar 3 fines de semana consecutivos}}

La restricción más compleja del modelo es la R5, que requiere el uso de la variable binaria auxiliar \( w_{ik} \). Esta regula cuántos fines de semana trabaja un mismo trabajador.

\[
w_{i,k} + w_{i,k+1} + w_{i,k+2} \le 2
\]

Su implementación en código es la siguiente:

\begin{lstlisting}[caption=Lógica de la Restricción R5]
# Restriccion R5: No trabajar 3 fines consecutivos
if num_fines >= 3:

    # Enlace entre las variables x e y (R5.1)
    # Se omite por brevedad, pero esta implementado en el solver.

    for i in range(num_trabajadores):
        for k in range(num_fines - 2):
            modelo += (
                w[i][k] + w[i][k+1] + w[i][k+2] <= 2,
                f"No3Fines_{i}_{k}"
            )
\end{lstlisting}

\vspace{0.4cm}
\subsection*{\textbf{Proceso de Ejecución Masiva}}

El archivo \texttt{ejecutar\_solver\_batch.py} recorre automáticamente los directorios:

\begin{center}
\texttt{data/small/}, \texttt{data/medium/}, \texttt{data/large/}
\end{center}

Para cada archivo JSON:

\begin{enumerate}
    \item Se carga la instancia.
    \item Se construye el modelo con PuLP.
    \item Se ejecuta el solver \texttt{lpsolve}.
    \item Se mide el tiempo de resolución.
    \item Se clasifica el resultado como óptimo, factible o infactible.
    \item Se guarda un archivo JSON en:
\end{enumerate}

\[
\texttt{./Entrega 2/resultados/<tamaño>/}
\]

Así, cada instancia genera su propio reporte independiente.

\vspace{0.4cm}
\subsection*{\textbf{Estructura del Módulo de Métricas}}

El sistema acumula información global para producir un archivo resumen:

\begin{lstlisting}[caption=Estructura de Recolección de Métricas]
resumen_total = {
    "fecha_ejecucion": datetime.now().isoformat(),
    "instancias_procesadas": 0,
    "instancias_optimas": 0,
    "instancias_infactibles": 0,
    "tiempo_total_segundos": 0,
    "resultados_por_tamano": {}
}
\end{lstlisting}

Este resumen incluye estadísticas por categoría (\texttt{small}, \texttt{medium}, \texttt{large}) y métricas agregadas.

\vspace{0.4cm}
\subsection*{\textbf{Ejemplo de Archivo de Resultados}}

A continuación se muestra un fragmento de un archivo JSON generado para una instancia \texttt{large}:

\begin{lstlisting}[language=json, caption=Ejemplo de Resultado JSON]
{
  "id_instancia": 11,
  "tipo": "large",
  "dias": 23,
  "trabajadores": 47,
  "estado": "Optimal",
  "valor_objetivo": 9362.0,
  "tiempo_resolucion_segundos": 0.179936,
  "factible": true,
  "asignaciones": [
    {
      "trabajador": 1,
      "dia": 1,
      "dia_nombre": "domingo_1",
      "turno": "noche",
      "disposicion": 8
    }
  ],
  "fecha_resolucion": "2025-11-09T02:00:57.060042"
}
\end{lstlisting}

\vspace{0.4cm}
\subsection*{\textbf{Notas Finales}}

\begin{itemize}
    \item El sistema está diseñado para ser escalable y completamente automatizado.
    \item Los tiempos de ejecución aumentan en instancias \texttt{large}, debido a la naturaleza entera del problema.
    \item Todos los archivos están documentados y explicados en \texttt{./Entrega 2/solver/}.
\end{itemize}
