\clearpage
\section{Generador de Instancias}

El generador fue implementado en \textbf{Python} y se encuentra en:

\begin{center}
\texttt{generador/src/Generador\_1\_Grupo2\_OPTI\_SJ.py}
\end{center}

Este script genera \textbf{cinco instancias por cada tamaño} (\textit{small}, \textit{medium}, \textit{large}), siguiendo los rangos establecidos en el enunciado. Las instancias se almacenan en carpetas separadas dentro de:

\begin{center}
\texttt{generador/instancias/}
\end{center}

El código fuente completo está disponible en el repositorio de GitHub \cite{generador2025}.

\vspace{0.5em}
Además, el repositorio incluye un script de chequeo que genera resúmenes estadísticos por instancia, útiles para el análisis posterior. Este script complementa el generador y se encuentra en el mismo repositorio citado.


\subsection*{Lógica de Generación y Formato de Salida}

\begin{itemize}
  \item Se definen aleatoriamente los días y trabajadores según el tipo de instancia:
  \begin{itemize}
    \item \textit{Small}: 5--7 días, 5--15 trabajadores, turnos \texttt{dia} y \texttt{noche}.
    \item \textit{Medium}: 7--14 días, 15--45 trabajadores, turnos \texttt{manana}, \texttt{tarde}, \texttt{noche}.
    \item \textit{Large}: 14--28 días, 45--90 trabajadores, mismos turnos que \textit{medium}.
  \end{itemize}

  \item Para cada día y turno, se genera una demanda \(r_{j,t}\) con una distribución normal truncada en cero, con media proporcional al número de trabajadores y turnos, y desviación estándar del 20\%. Luego:
  \begin{enumerate}
    \item Se limita la demanda por la disponibilidad real del turno (trabajadores con disposición positiva).
    \item Si la suma diaria excede la capacidad máxima (\( \text{max\_turnos\_dia} \times \text{trabajadores}\)), se escala proporcionalmente.
  \end{enumerate}

  \item Para cada trabajador, día y turno, se genera una disposición \(c_{i,j,t} \sim \mathcal{U}\{0,10\}\).

  \item Cada instancia se guarda en dos formatos:
  \begin{itemize}
    \item \texttt{.json}: contiene \texttt{id\_instancia}, \texttt{tipo}, \texttt{dias}, \texttt{trabajadores}, \texttt{demanda\_dias} y \texttt{disposicion}.
    \item \texttt{.csv}: archivo plano con las disposiciones individuales, con columnas:
\begin{center}
\texttt{trabajador, dia, turno, disposicion}
\end{center}
Este formato permite inspeccionar rápidamente las preferencias de cada trabajador, pero no incluye la demanda ni otros metadatos.
  \end{itemize}
\end{itemize}

\clearpage
\subsection*{Descripción del Código}

El script realiza dos tareas principales:

\begin{enumerate}
  \item \textbf{Inicialización:} Elimina las carpetas existentes en \texttt{../instancias/} para evitar duplicados y crea nuevas subcarpetas para cada tipo de instancia.
  \item \textbf{Generación:} Para cada tipo de instancia, se generan 5 archivos con datos aleatorios. Se utiliza una semilla aleatoria (\texttt{random.seed()}) para permitir variabilidad entre ejecuciones. El proceso incluye:
  \begin{itemize}
    \item Asignación de días y trabajadores dentro de los rangos definidos.
    \item Generación de nombres de días con formato \texttt{lunes\_1}, \texttt{martes\_1}, etc.
    \item Cálculo de demanda por turno usando una distribución normal truncada en cero.
    \item Ajuste de demanda por disponibilidad y escalado por capacidad diaria.
    \item Generación de disposiciones por trabajador, día y turno.
    \item Escritura de archivos \texttt{.json} y \texttt{.csv}.
  \end{itemize}
\end{enumerate}

\subsection*{Fragmento del Código}

\begin{lstlisting}[language=Python, basicstyle=\ttfamily\footnotesize, breaklines=true, caption={Generador de instancias con control de capacidad}]
def generar_instancias(max_turnos_dia=2, dispo_umbral=0):
    random.seed()  # Semilla aleatoria para variabilidad
    semana_dias = ["lunes", "martes", "miercoles", "jueves", "viernes", "sabado", "domingo"]

    tipos = ["small", "medium", "large"]
    dias_rangos = [(5, 7), (7, 14), (14, 28)]
    trabajadores_rangos = [(5, 15), (15, 45), (45, 90)]
    turnos_rangos = [["dia", "noche"], ["manana", "tarde", "noche"], ["manana", "tarde", "noche"]]

    id_instancia = 1
    for tipo, dias_rng, trab_rng, turnos in zip(tipos, dias_rangos, trabajadores_rangos, turnos_rangos):
        for i in range(5):  # 5 instancias por tipo
            dias = random.randint(*dias_rng)
            trabajadores = random.randint(*trab_rng)
            cantidad_turnos = len(turnos)
            # Generacion de disposicion y demanda omitida por brevedad
\end{lstlisting}