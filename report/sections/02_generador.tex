\clearpage
\section{Generador de Instancias}

El generador fue implementado en \textbf{Python} y se encuentra en:

\begin{center}
\texttt{generador/src/Generador\_1\_Grupo2\_OPTI\_SJ.py}
\end{center}

Este script genera \textbf{cinco instancias por cada tamaño} (\textit{small}, \textit{medium}, \textit{large}), siguiendo los rangos establecidos en el enunciado. Las instancias se almacenan en carpetas separadas dentro de:

\begin{center}
\texttt{generador/instancias/}
\end{center}

El código fuente completo está disponible en el repositorio de GitHub \cite{generador2025}.

\subsection*{Lógica de Generación y Formato de Salida}

\begin{itemize}
  \item Se definen aleatoriamente los días y trabajadores según el tipo de instancia:
  \begin{itemize}
    \item \textit{Small}: 5–7 días, 5–15 trabajadores, turnos \texttt{día} y \texttt{noche}.
    \item \textit{Medium}: 7–14 días, 15–45 trabajadores, turnos \texttt{mañana}, \texttt{tarde}, \texttt{noche}.
    \item \textit{Large}: 14–28 días, 45–90 trabajadores, mismos turnos que \textit{medium}.
  \end{itemize}
  
  \item Para cada día y turno, se genera una demanda \( r_{j,t} \) con una \textbf{distribución normal truncada}, con media proporcional al número de trabajadores y turnos, y desviación estándar del 20\%. Esto permite simular variabilidad en la carga de trabajo diaria.

  \item Para cada trabajador, día y turno, se genera una disposición \( c_{i,j,t} \sim \mathcal{U}\{0,10\} \), donde 0 indica que el trabajador no puede realizar ese turno. Esta variabilidad permite representar preferencias y restricciones personales.

  \item Cada instancia se guarda en dos formatos:
  \begin{itemize}
    \item \texttt{.json}: contiene la estructura completa de la instancia, incluyendo:
    \begin{itemize}
      \item \texttt{meta}: metadatos como el tipo de instancia, número de días y trabajadores.
      \item \texttt{sets}: definición de los conjuntos \( T \), \( D \), \( S \).
      \item \texttt{demand}: matriz con los valores de \( r_{j,t} \).
      \item \texttt{preferences}: matriz tridimensional con los valores de \( c_{i,j,t} \).
    \end{itemize}
    \item \texttt{.csv}: archivo plano con las disposiciones individuales, útil para inspección rápida o visualización.
  \end{itemize}
\end{itemize}
\clearpage
\subsection*{Descripción del Código}

El script realiza dos tareas principales:

\begin{enumerate}
  \item \textbf{Inicialización:} Elimina las carpetas existentes en \texttt{generador/instancias/} para evitar duplicados y crea nuevas subcarpetas para cada tipo de instancia.
  \item \textbf{Generación:} Para cada tipo de instancia, se generan 5 archivos con datos aleatorios. Se utiliza una semilla fija (\texttt{random.seed(1234)}) para asegurar reproducibilidad. El proceso incluye:
  \begin{itemize}
    \item Asignación de días y trabajadores dentro de los rangos definidos.
    \item Generación de nombres de días con formato \texttt{lunes\_1}, \texttt{martes\_1}, etc., para facilitar la identificación de semanas.
    \item Cálculo de demanda por turno usando una distribución normal con media ajustada dinámicamente.
    \item Generación de disposiciones por trabajador, día y turno, con valores enteros entre 0 y 10.
    \item Escritura de archivos \texttt{.json} y \texttt{.csv} con la información generada.
  \end{itemize}
\end{enumerate}

\subsection*{Fragmento del Código}

\begin{lstlisting}[language=Python, basicstyle=\ttfamily\footnotesize, breaklines=true]
def generar_instancias():
    random.seed(1234)
    tipos = ["small", "medium", "large"]
    dias_rangos = [(5, 7), (7, 14), (14, 28)]
    trabajadores_rangos = [(5, 15), (15, 45), (45, 90)]
    turnos_rangos = [["dia", "noche"], ["manana", "tarde", "noche"], ["manana", "tarde", "noche"]]
    for tipo, dias_rng, trab_rng, turnos in zip(tipos, dias_rangos, trabajadores_rangos, turnos_rangos):
        for _ in range(5):
            dias = random.randint(*dias_rng)
            trabajadores = random.randint(*trab_rng)
            cantidad_turnos = len(turnos)
            demanda_dias = {}
            disposicion = []
            for j in range(dias):
                mu = (trabajadores / cantidad_turnos) * random.uniform(1.0, 1.2)
                sigma = mu * 0.2
                demandas_turnos = {t: max(0, int(random.normalvariate(mu, sigma))) for t in turnos}
                demanda_dias[f"dia_{j+1}"] = demandas_turnos
            for i in range(1, trabajadores + 1):
                for d in range(1, dias + 1):
                    for t in turnos:
                        dispo = random.randint(0, 10)
                        disposicion.append({"trabajador": i, "dia": d, "turno": t, "disposicion": dispo})
            # Escritura de archivos omitida por brevedad
\end{lstlisting}

El código completo incluye la escritura de archivos y está documentado en el repositorio citado.

