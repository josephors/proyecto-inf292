\section{Tiempos de resolución y escalamiento}
Medimos el tiempo de resolución total y por instancia usando CBC. Todas las instancias se resuelven en menos de \SI{0.3}{s} (máximo observado); el total de las 15 instancias fue \(\approx\) \SI{1.31}{s}. Los resultados muestran un crecimiento aproximadamente cuadrático con el tamaño (Figuras~\ref{fig:tiempos} y \ref{fig:tiempos-escalamiento}).

\begin{figure}[H]
	\centering
	\begin{subfigure}{0.49\textwidth}
		\centering
		\includegraphics[width=\linewidth]{../Entrega 2/analisis/graficos/tiempos_vs_tamano.png}
	\end{subfigure}\hfill
	\begin{subfigure}{0.49\textwidth}
		\centering
		\includegraphics[width=\linewidth]{../Entrega 2/analisis/graficos/tiempos_vs_variables.png}
	\end{subfigure}
	\caption{Tiempo de resolución por tamaño y relación con número de variables.}
	\label{fig:tiempos}
\end{figure}

\begin{figure}[H]
	\centering
	\begin{subfigure}{0.49\textwidth}
		\centering
		\includegraphics[width=\linewidth]{../Entrega 2/analisis/graficos/tiempos_promedio_tipo.png}
	\end{subfigure}\hfill
	\begin{subfigure}{0.49\textwidth}
		\centering
		\includegraphics[width=\linewidth]{../Entrega 2/analisis/graficos/tiempos_comparacion_escalabilidad.png}
	\end{subfigure}
	\caption{Promedio por tipo y comparación de escalabilidad (log--log).}
	\label{fig:tiempos-escalamiento}
\end{figure}

En términos prácticos, la herramienta es interactiva para el tamaño del problema presentado, y la tendencia sugiere que podemos manejar instancias mayores con tiempos todavía aceptables, especialmente aplicando límites de tiempo o estrategias de arranque si fuese necesario.